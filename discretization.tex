\todo{work on this after the new pruducts structure is defined}
The discretization product contains several packages to handle discretizations of differential equations.

\subsection{Intrepid2}
\todo{Nate, let's coordinate on what to put here}
Intrepid2 provides interoperable tools for compatible discretizations of PDEs. Intrepid2 mainly focus on local assembly of continuous and discontinuous finite elements, and provides tools for finite volume discretizations as well. The present version of Intrepid2 implements compatible finite element spaces of orders up to 10 for H(grad), H(curl), H(div) and L2 function spaces on frequently used elements such as triangles, quadrilaterals, tetrahedrons and hexahedrons. It provides both Lagrangian basis functions and Hierarchical basis functions and it implements sevral performance optimization (sum factorizations) explointing underline structure of the problem (e.g. tensor-product elements or other symmetries). Intrepid2 provides orientation tools for matching the degrees of freedom on shared edges and faces. It also provides projection tools for projecting functions in H(grad), H(curl), H(div) and L2 to the respective discrete spaces. Intrepid2 achieves performance portability using the Kokkos programming model. 

\subsection{Phalanx}
\todo{Roger, please edit/expand}
The package is a local field evaluation kernel specifically designed for general partial differential equation solvers. The main goal of Phalanx is to decompose a complex problem into a number of simpler problems with managed dependencies to support rapid development and extensibility of the PDE code. Through the use of template metaprogramming concepts, Phalanx supports arbitrary user defined data types and evaluation types. This allows for unprecedented flexibility for direct integration with user applications and provides extensive support for embedded technology such as automatic differentiation for sensitivity analysis and uncertainty quantification.

\subsection{Panzer}
\todo{Roger, please edit/expand}
The package provides global tools for finite element analysis. It handles continuous and discontinuous high-order compatible finite elements, as implemented in Intrepid2 on unstructured meshes. Panzer relies on Phalanx to manage with efficiency and flexibility the assembly of complex problems. Panzer also enables the solution of nonlinear problems, by interfacing with several Trlinos linear and nonlinear solvers. It computes derivatives and sensitivities through automatic differentiation (Sacado). It supports both Epetra and Tpetra data structures and achieves performance portability through the Kokkos programming model.

\subsection{Compadre}
The Compadre package provides tools for the approximation of linear operators (including point evaluation and derivatives), given the location of samples of a function over an unstructured cloud of points. The resulting stencils, when applied directly to samples of the function at these locations, provides an approximation of the linear operator acting on the function at the point(s) queried. This is useful for meshed and meshless data transfer applications. Values of the function at the specified locations can also be viewed as unknowns, in which case the solution returned by Compadre can be used as a stencil for meshless discretization of PDEs. 

The package uses generalized moving least squares (GMLS) for approximating functionals. We plan on implementing other meshless methods like radial basis functions in the future. Compadre achieves performance portability by using hierarchical parallelism with the Kokkos programming model. A brief description of GMLS follows, but technical details can be found in \cite{mirzaei2012generalized,wendland2004scattered}.

Consider $\phi$ of function class $\mathbf{V}$. Consider a collection of samples $\Lambda = \left\{\lambda_i(\phi)\right\}_{i=1}^{N}$ corresponding to a quasi-uniform\cite{wendland2004scattered} collection of data sites $\mathbf{X}_h = \left\{\mathbf{x}_i\right\} \subset \mathbb{R}^d$ characterized by fill distance $h$. To approximate a given linear target functional $\tau_{\tilde{x}}$ associated with a target site $\tilde{x}$, we seek a reconstruction $p \in \mathbf{V}_h$, where $\mathbf{V}_h \subset \mathbf{V}$ is a finite dimensional space chosen to provide good approximation properties, with basis $\mathbf{P}=\left\{P\right\}_{i=1}^{dim(V_h)}$. We perform this reconstruction in the following weighted $\ell_2$ sense:

\begin{equation}
\label{gmls}
p = \underset{{q \in \mathbf{V}_h}}\argmin \sum_{i=1}^N \left( \lambda_i(\phi) -\lambda_i(q) \right)^2 \omega(\lambda_i,\tau_{\tilde{x}})
\end{equation}
where $\omega$ is a locally supported positive function. Compadre offers a set of choices for to $\omega = \Phi(|\tilde{x}-\mathbf{x}_i|)$, where $|\cdot|$ denotes the Euclidean norm an

With the optimal reconstruction $p$ computed, the target functional is approximated via $\tau_{\tilde{x}} (\phi) \approx \tau^h_{\tilde{x}} (\phi) := \tau_{\tilde{x}} (p)$. As an unconstrained $\ell_2$-optimization problem, this process admits the explicit form


\begin{equation}
\label{discreteTarget}
\tau^h_{\tilde{x}}(\phi) = \tau_{\tilde{x}}(\mathbf{P})^\intercal \left(\Lambda(\mathbf{P})^\intercal \mathbf{W} \Lambda(\mathbf{P})\right)^{-1} \Lambda(\mathbf{P})^\intercal \mathbf{W} \Lambda(\phi),
\end{equation}
where we denote:
\begin{itemize}
  \item $\tau_{\tilde{x}}(\mathbf{P}) \in \mathbb{R}^{dim(V_h)}$ is a vector with components consisting of the target functional applied to each basis function.
  \item $\mathbf{W} \in \mathbb{R}^{N \times N}$ is a diagonal matrix with diagonal entries consisting of $\left\{\omega(\lambda_i,\tau_{\tilde{x}})\right\}_{i=1,...,N}$.
  \item $\Lambda(\mathbf{P}) \in \mathbb{R}^{N \times dim(V_h)}$ is a rectangular matrix whose $(i,j)$ entry corresponds to the application of the $i^{th}$ sampling functional applied to the $j^{th}$ basis function.
  \item $\Lambda(\phi) \in \mathbb{R}^N$ is a vector consisting of the $N$ samples of the function $\phi$.
\end{itemize}
We note that by taking the contraction of the tensors appearing in Equation \ref{discreteTarget} and exploiting the compact support of $\omega$, we may interpret the output of the GMLS process as a finite difference-like stencil of the form 
\begin{equation}
\tau^h_{\tilde{x}}(\phi) = \sum_{\mathbf{x}_i \in B^\epsilon(\tilde{x})} \alpha_i \lambda_i(\phi),
\end{equation}
where $B^\epsilon(\tilde{x})$ denotes the $\epsilon$-ball neighborhood of the target site $\tilde{x}$. Therefore, GMLS admits an interpretation as an automated process for generating generalized finite difference methods on unstructured point clouds. The computational cost of solving the GMLS problem amounts to inverting a small linear system which may be assembled using only information from neighbors within the support of $\omega$, and construction of such stencils across the entire domain is embarrassingly parallel.

Compadre allows users to control the weighting kernel ($\omega$), degree of the reconstruction basis ($\mathbf{V}_h$), the sampling functionals ($ \left\{\lambda_i(\phi)\right\}_{i=1}^{N}$), and the target operator ($\tau_{\tilde{x}}$); this allows control over smoothness of the reconstruction, locality of the resulting stencil, order of accuracy of the reconstruction (assuming regularity of the function embedded in the point cloud data), choice of what the sampled data or degrees of freedom represent, and linear operator action, respectively.
