
\todo{Should we add some sentences on ``how to contribute"?}

\todo{@Siva: please write this!}

\subsection{Contributing to Trilinos}

Contributions to Trilinos can be offered through the standard GitHub pull request model.
Proposed code changes are required to pass a set of tests as well as review and approval prior to be merged.
Detailed instructions can be found in the contributing guidelines in the source code repository.

\subsection{Platforms for exchange among users or developers}

The community of Trilinos users and developers operates several forums for exchange and discussion.
Technical discussions about the source code and its development happen within the Trilinos GitHub repository\footnote{\url{https://github.com/trilinos/Trilinos}}.
Several mailing lists (see \url{https://trilinos.github.io/mail_lists.html}) distribute relevant information and updates on Trilinos.

For in-person exchange, the \emph{Trilinos User-developer Group (TUG) Meeting} takes place at Sandia National Laboratories in Albuquerque every year.
At TUG, all Trilinos users and developers can come together to inform themselves about recent progress and advances,
discuss current challenges and upcoming topics relevant to the entire Trilinos community.

The \emph{European Trilinos User Group (EuroTUG) Meeting} series\footnote{\url{https://eurotug.github.io}}
offers a platform for Europe-based users and developers of the Trilinos project.
EuroTUG facilitates easy access to the Trilinos community and reduced travel burdens for Europe-based researchers and application engineers
who are interested in the Trilinos project.
It covers tutorial sessions to educate the community,
user presentations to demonstrate capabilities and features of various application codes using Trilinos,
and updates from developers to spread news and ongoing work to all interested parties.
