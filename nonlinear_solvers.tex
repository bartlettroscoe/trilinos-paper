
The Nonlinear Analysis product area provides the top level algorithms for a computational simulation or design study.
These include nonlinear solvers, time integration, bifurcation tracking, stability analysis, parameter continuation, optimization, and uncertainty quantification.
A common theme of this collection is the philosophy of ``analysis beyond simulation'', which aims to automate many computational tasks that are often performed by application code users by trial-and-error or repeated simulation.
Tasks that can be automated include performing parameter studies, sensitivity analysis, calibration, optimization, time step size control, and locating instabilities.
This capability area additionally includes utilities for the nonlinear analysis. These include automatic differentiation tools that can provide the derivatives critical to the analysis algorithms and the abstraction layers and interfaces for application callbacks.

\subsection{Thyra}
The Thyra library provides abstractions for use in writing high level abstract numerical algorithms (ANAs). Thyra is used by other Trilinos packages for implementing ANAs such as solvers for linear equations, nonlinear equations, bifurcation and stability analysis, optimization and uncertainty quantification. The abstractions hide concrete linear algebra and solver implementations so that the algorithms can be written once and reused across many application codes. For example, one could write a Krylov solver and it could be used with multiple concrete linear algebra implementations such as Tpetra in Trilinos, Epetra in Trilinos or the vector/matrix objects in PETSc.

Thyra is organized into three sets of abstractions depending on the type of operation required. All are templated on the scalar type. The sets are:
\begin{itemize}
\item \emph{Vector/Operator Interfaces:} The first level is an abstraction for vectors and operators \cite{Bartlett2007}. This set includes abstractions for vector spaces, vectors, multivectors and linear operators. Thyra provides a default set of vector/vector and vector-operator operations for standard linear algebra tasks such ad AXPY from BLAS. Defining an extensible interface such that users can add arbitrary vector/matrix operations and maintain performance can be difficult. A general tool to achieve this is the Trilinos utility package called RTOp \cite{rtop}. RTOp provides an interface for users to add new Thyra operations. RTOp accepts a vector of Thyra input objects and a vector of Thyra output objects along with a functor. The RTOp will apply the functor to each element of the data.

\item \emph{Operator Solve Interfaces:} The second set of abstractions supply abstract linear solvers that can be attached to operators. The interfaces include base classes for preconditioners, preconditioner factories, linear operators with a solver, and factories for linear operators with a solver. The factories build and update the preconditioners and linear solvers as needed. The solver interfaces allow one to wrap any concrete linear solver or preconditioner for use with the ANAs. Trilinos provides a separate library named Stratimikos that wraps all of the Trilinos preconditioners and linear solvers in Thyra abstractions. This library is driven by the Teuchos::ParameterList options.

\item \emph{Nonlinear Interfaces:} The final set of abstractions support nonlinear operations. Thyra provides an interface that wraps nonlinear solvers. Implicit time integration and optimization solvers typically require nonlinear solves and can abstract away the concrete implementation via Thyra. The NOX package in Trilinos provides a concrete implementation of this interface. The final interface in Thyra is for querying a physics application for common objects required by high level Newton-based solvers. The Thyra::ModelEvaluator interface allows codes to ask the application to evaluate residuals, Jacobians, parametric sensitivities, Hessians and post processed quantities/derivatives of interest. The interface is designed to be stateless (i.e. consecutive calls with the same inputs should produce the same outputs). This design was chosen to avoid pitfalls in complex application codes interacting with general nonlinear algorithms. The design does not preclude its use in stateful applications that have a ``memory'' between time steps, but rather forces the explicit use of the stateful information in the interface.
\end{itemize}

\subsection{NOX}
NOX, short for Nonlinear Object-oriented Solutions, provides robust and efficient algorithms for solving sets of nonlinear equations.
NOX implements a number of Newton-based globalization techniques including line search \cite{Pawlowski2006}, trust region \cite{Pawlowski2006,Pawlowski2008} and homotopy algorithms \cite{Coffey2003}.
Additionally, it provides lower and higher order models including Broyden, Anderson acceleration \cite{Walker2011} and tensor methods \cite{Bader2005}.
The algorithms have been designed for large scale parallel inexact linear solvers using Krylov methods and supports Jacobian-free Newton-Krylov variants.
The library interacts with application codes through the Thyra model evaluator interface.
NOX provides linear algebra abstractions for applications to use custom implementations with the algorithms.
The library additionally contains abstractions for solvers, directions and line searches that allow users to further customize the algorithms.
NOX provides various stopping criteria including absolute, relative and weighted root mean square norms, as well as stagnation and NaN detection.
Applications can build custom stopping criteria within a tree-based logical structure and provide additional criteria through the StatusTest abstraction.

\subsection{LOCA}
LOCA~\cite{Salinger2005}, short for the Library of Continuation Algorithms, provides techniques for computing families of solutions to nonlinear equations as well as methods for investigating their stability when these nonlinear equations define equilibria of dynamical systems.  It builds on the NOX nonlinear solver package to track solutions to sets of nonlinear equations as a function of one or more parameters (continuation).  Given an interface to NOX defining the nonlinear equations, all users must additionally provide is an ability to set the parameter values that are being varied.  LOCA provides several continuation methods, including pseudo-arclength continuation which allows tracking solution curves around turning points/folds.  Furthermore, LOCA has hooks to call the Anasazi eigensolver package to estimate leading eigenvalues of the linearization at each point along the continuation curve for linear stability analysis, including various spectral transformations highlighting eigenvalues in different regimes (e.g., largest magnitude, largest real, ...).  Finally, LOCA implements equations augmenting the original nonlinear equations to locate and track bifurcation points where linear stability changes (e.g., turning point, pitchfork, and Hopf bifurcations) as a function of additional parameters. Examples of using LOCA for stability and bifurcation analysis include stability of a differentially heated cavity \cite{Salinger2002}, design of chemical vapor deposition reactors \cite{Pawlowski2001}, flow instabilities in counterflowing jets \cite{pawlowski_salinger_shadid_mountziaris_2006} and ignition in laminar diffusion flames \cite{Shadid20061846}.

\subsection{Tempus}

Tempus (Latin meaning time as in “tempus fugit” -> “time flies”)
is the Trilinos time-integration package for advanced transient
analysis.  It includes various time integrators and embedded
sensitivity analysis for next-generation code architectures.  Tempus
provides “out-of-the-box” time-integration capabilities, which
allows users to quickly and easily incorporate time-integration
capabilities to their applications and switch between various time
integrators depending on the simulation needs.  Additionally, Tempus
provides “build-your-own” capabilities, which allows applications
to incorporate various Tempus components to augment or replace
application transient capabilities. Other capabilities include
embedded error analysis, sensitivity analysis, transient optimization
with ROL.

Tempus provides a general infrastructure for the time evolution of
solutions through a variety of general integration schemes.  Tempus
provides time integrators for explicit and implicit methods and for
first- and second-order ODEs.  It can be used from small systems of
equations (e.g., single ODEs for the time evolution of plasticity
models, and multiple ODEs for coupled chemical reactions) to
large-scale transient simulations requiring exascale computing
(e.g., flow fields around reentry vehicles and magneto-hydrodynamics).

Tempus has several components that can be used in concert or
individually, depending on the needs of the application.
\begin{itemize}
  \item Integrators are the time-loop structure for time integration
  and provide several features, e.g., control the advancement of
  the solution, selection of the next timestep size and solution
  output.

  \item Time Steppers are individual methods that advance the
  solution from one step to the next.  A variety of time steppers
  are available:
  \begin{itemize}
    \item Classic one-step methods (e.g., Forward Euler and Trapezoidal
    Method)
    \item Explicit Runge-Kutta methods (e.g., RK Explicit 4 Stage)
    \item Diagonally Implicit Runge-Kutta (DIRK) Methods (e.g.,
    general tableau DIRK and many specific DIRK/SDIRK methods)
    \item Implicit-Explicit (IMEX) Runge-Kutta Methods (e.g., IMEX
    RK SSP2, IMEX RK SSP3, and general tableau IMEX RK methods)
    \item Multi-Step Methods (i.e., BDF2)
    \item Second-order ODE Methods (e.g., Leapfrog, Newmark methods
    and HHT-Alpha)
    \item Steppers with subSteppers (e.g., operator-split and
    subcycling methods)
  \end{itemize}

  \item Solution History is used to maintain the solution during
  time-step failure, solution restart/output, interpolation of
  solution between time steps, and to provide the solution for
  transient adjoint sensitivities.

  \item Timestep Control and Strategies provide methods to select
  the time-step size based on user input and/or temporal error
  control (e.g., bounding min/max time-step size, relative/absolute
  maximum error, and timestep adjustments for output and checkpointing)

\end{itemize}
Additionally, Tempus has several mechanisms which allow users to
insert application-specific algorithms into Tempus components (e.g.,
through observers and creating derived classes).

\subsection{Piro}
%Piro~\cite{osti_1231283} is the top-level, unifying package for nonlinear analysis. 
%The main purpose of the package is to provide driver classes for the common uses of Trilinos nonlinear analysis tools. 
%These drivers all can be constructed similarly, with a \lstinline{Thyra::ModelEvaluator} and a \lstinline{Teuchos::ParameterList}, to make it simple to switch between different types of analysis. 
%They also all inherit from the same base classes (response-only model evaluators) so that the resulting analysis can in turn be driven by non-intrusive analysis routines.
%The Piro library is a new package that is part of the Trilinos framework. Piro is a convenience package provides a simple and unified interface to many of the solver and analysis packages in Trilinos. The Piro package wraps many of the common usages of other existing Trilinos packages (e.g. NOX, Rythmos, TriKota) and provides a simple interface. In this way, a computer simulation code that wants to use Trilinos analysis tools can access them with just and handful of lines of Piro code. This saves much effort in interfacing their code to each of the other codes one at a time.
Piro~\cite{osti_1231283} provides a simple and unified interface to nonlinear analysis tools in Trilinos, providing capabilities for computing sensitivities, performing parameter continuation and bifurcation analysis, and solving constrained optimization problems.
%The main purpose of the package is to provide driver classes for the common uses of Trilinos nonlinear analysis tools.
In particular, Piro implements main driver classes for:
\begin{itemize}
	\item \emph{Linear/nonlinear solvers and sensitivity analysis} for transient and nontransient problems. As an example, this capability can be used to compute the discrete solution $u$ of a partial differential equation depending on some parameter $p$, and compute the sensitivity of a quantity of interest of the solution with respect to the parameter $p$. Sensitivities can be computed in a forward or adjoint fashion, the latter being preferable in presence of high-dimensional parameters. This capability relies on NOX for the solution of nonlinear problems and Tempus for transient problems.
	\item \emph{Constrained optimization problems} with linear/nonlinear equality constraints. Piro provide tools for transient and nontransient (snapshot) optimizations, featuring gradient-based reduced-space and full-space methods. This capability is used by applications such as Albany, to perform large-scale PDE-constrained optimization problems \cite{Perego2022}. Piro interfaces with ROL for algorithms to solve constrained optimization problems.
\item \emph{Parameter Continuation and Bifurcation analysis}. This capability is provided through the package LOCA.
\end{itemize}
These driver classes share a similar interface based on the \texttt{Thyra::ModelEvaluator} and the \texttt{Teuchos::ParameterList} classes. Applications define the problems to be targeted (e.g., the equations to be solved, the parameters, the quantities of interests) by providing a concrete implementations of the \texttt{Thyra::ModelEvaluator}, see \cite{pawlowski2012automating,pawlowski2012automatingpart2}.


\subsection{ROL}
Rapid Optimization Library (ROL) \cite{rol,ROL2022ICCOPT} is the
Trilinos library for numerical optimization. ROL brings an extensive
collection of state-of-the-art optimization algorithms to virtually
any application. Its programming interface supports any computational
hardware, including heterogeneous many-core systems with digital and
analog accelerators.
ROL has been used with great success for optimal control, optimal design,
inverse problems, image processing and mesh optimization, in application
areas including geophysics, climate science \cite{Perego2022}, structural
dynamics \cite{AQUINO2019323,BUNTING2021107295}, fluid dynamics
\cite{Antil2023}, electromagnetics, quantum computing, hypersonics and
geospatial imaging \cite{Kouri2014}.

The key design, performance and algorithmic features of ROL can be
summarized as follows:

\begin{itemize}
\item
\emph{Vector abstractions and matrix-free interface for universal applicability}.
Similar to other Trilinos packages that implement abstract numerical
algorithms, ROL's design is centered around an abstract linear algebra
interface, through the {\tt ROL::Vector} class, which enables the use of any
ROL~algorithm with any data type, such as the C++ {\tt std::vector}, MPI-parallel
data structures (e.g., Epetra, Tpetra, PETSc, HYPRE vectors), GPU-supporting data
structures (e.g., Kokkos, ArrayFire \cite{Yalamanchili2015}), etc.  The abstract
linear algebra interface further enables the definition of custom vector dot
products, which are critical for performance in large-scale applications.
Finally, all ROL algorithms are matrix-free and rely on user-defined
applications of linear and nonlinear operators.  ROL's vector class design is
inspired by the Hilbert Class Library \cite{hcl}, the Rice Vector Library \cite{rvl}
and the RTOp framework \cite{rtop}, while ROL's operator interfaces for
simulation-based optimization, known as the SimOpt middleware,
are related to \cite{Heinkenschloss1999}.
\item
\emph{Modern algorithms for unconstrained and constrained optimization}.
ROL features a large collection of well-established algorithms for smooth
optimization as well as several novel algorithms developed by the ROL team.
ROL categorizes optimization problems into four basic types, based on the
presence of certain constraints: Type U (unconstrained), Type B (bound
constraints), Type E (equality constraints) and Type G (general constraints).
Additionally, each problem type can include linear constraints, which ROL can
reduce or eliminate algorithmically. The novel algorithms include augmented
Lagrangian methods for infinite-dimensional problems with general constraints
\cite{ALESQP} and matrix-free trust-region methods for convex-constrained
optimization \cite{Kouri2022}.
\item
\emph{Easy-to-use methods for stochastic and risk-aware optimization}.
ROL contains specialized interfaces and algorithms for stochastic optimization.
Optimization problems with stochastic and uncertain problem data pose challenges
in problem formulation and efficient numerical solution methods.  ROL's interface
for stochastic optimization enables the use of numerous modern risk measures,
probabilistic functions and robust problem formulations. To approximate
risk-averse problems \cite{shapiro2021lectures}, ROL includes sample-based
approaches such as sample average approximation, stochastic approximation, and
adaptive sparse-grid quadrature \cite{kouri2013trust,kouri2014inexact}. ROL
also includes specialized algorithms such as progressive hedging
\cite{rockafellar1991scenarios} and the primal-dual risk minimization algorithm
\cite{kouri2022primal}.
\item
\emph{Fast and robust algorithms for nonsmooth optimization}.
ROL includes specialized algorithms to minimize the sum of a smooth function
and a nonsmooth, convex function.  This class of problems is ubiquitous in
computational science.  For example, nonsmooth sparsifying regularizers such as
the $\ell^1$-norm are common in data science and machine learning.  These
algorithms can be used to solve convex constrained optimization problems.
ROL's methods for this class of problems are built on the proximity operator
and include proximal gradient and Newton methods
\cite{beck2017first,kanzow2021globalized,ochs2014iPiano}.
As discussed in the subsequent bullet, ROL can also incorporate inexact
values and gradients for the smooth objective function term using trust regions
\cite{baraldi2023proximal}.
ROL also includes the trust bundle method \cite{schramm1992bundle}
for solving general nonconvex, nonsmooth optimization problems.
\item
\emph{Trust-region methods for inexact and adaptive computations}.
ROL controls inexact function evaluations and exploits adaptive computations
and multi-fidelity models through its trust-region methods. Several of these methods
have been pioneered by the ROL team.  Their implementations typically require some
knowledge of error in the computations, i.e., an \emph{error estimate}.
For a comprehensive overview of problem formulations, algorithms with error control,
and their ROL implementations, see \cite{Kouri2018}.
\item
\emph{PDE-OPT application development kit for PDE-constrained optimization}.
To build and optimize models based on partial differential equations (PDEs),
ROL offers an application development kit, called PDE-OPT. PDE-OPT is based on a
modular design with three layers: local finite element computations,
which encode the physical equations governing the optimization problem;
global computations, which utilize Tpetra data structures;
and an interface to ROL, enabling efficient optimization.
To solve PDE-constrained optimization problems, the users need only implement the
local finite element layer of PDE-OPT, i.e., the evaluations of weak forms on
a mesh element; PDE-OPT automates all remaining steps.
\end{itemize}



\subsection{Sacado}

Sacado \cite{SacadoURL,phipps2012efficient,phipps2008large} provides forward and reverse-mode operator overloading-based automatic differentiation (AD) tools within Trilinos.
%The package provides both forward and reverse-mode AD data types. 
Sacado forward AD tools have been integrated into Kokkos and have demonstrated good performance on GPU architectures~\cite{phipps2022automatic}.
Sacado, along with its Kokkos integration, provides high-performance derivative capabilities to numerous Office of Science and NNSA extreme scale applications, including Albany for solid mechanics and land ice modeling~\cite{Salinger2016,MPASAlbany2018}, 
Charon for semiconductor device modeling~\cite{CharonUsersManual2020} and multiphase chemically reacting flows~\cite{Musson2009}, Drekar for computational fluid dynamics (CFD)~\cite{Sondak2021,Shadid2016}, magnetohydrodynamics~\cite{Shadid2016mhd} and 
plasma physics~\cite{Crockatt2022,Miller2019}, Xyce for electronic circuit simulation~\cite{xyceTrilinos,xycePCE}, and SPARC for hypersonic fluid flows~\cite{SparcValidation}. 

\subsection{Stokhos}

Stokhos~\cite{phipps2015stokhos,Phipps2016,phipps2014exploring} provides implementations of two intrusive uncertainty quantification strategies: 
the intrusive stochastic Galerkin uncertainty quantification method~\cite{ghanem1990polynomial,ghanem2003stochastic} and the embedded ensemble propagation~\cite{phipps2017embedded}.

For the first one, Stokhos provides methods for computing intrusive stochastic Galerkin projections such as Polynomial Chaos and Generalized Polynomial Chaos, 
interfaces for forming the resulting nonlinear systems, and linear solver methods for solving block stochastic Galerkin linear systems.
The implementation targets GPU performances using Kokkos and by commuting the layout of the Galerkin operator to be outer-spatial and inner-stochastic~\cite{phipps2014exploring}.
The stochastic Galerkin implementation of Stokhos has been used in~\cite{constantine2014efficient} to efficiently propagate uncertainty in multiphysics systems by reducing the full system with a nonlinear elimination method.

The embedded ensemble propagation consists in propagating a subset of samples gathered into a so-called ensemble through the forward simulation at once.
It builds on~\cite{pawlowski2012automating} for automating embedded analysis capabilities; Stokhos defines an ensemble type, a SIMD data type, that is able to store
the values of the input, output, and state variables for every sample of an ensemble. This type can then be used in the Tpetra solver stack as a template argument for the scalar type.
This approach allows to save computation time in four ways: the sample-independent data and computation can be reused for every sample of an ensemble, the memory access pattern is improved,
the operations on the ensemble type can be vectorized efficiently, and the message passing costs are reduced by sending fewer but larger messages.
However, the approach requires solvers and BLAS functions to be aware of the extra dimension associated to the ensemble; for example, a GMRES for ensemble types~\cite{liegeois2020gmres} needs to monitor 
the convergence of the individual sample in order to decide when to stop based on the union of the information.
